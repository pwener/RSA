\chapter[Considerações Finais]{Considerações Finais}
\label{chap:consideracoes}
	Neste trabalho procuramos esclarecer ao leitor os fundamentos da criptografia, exibindo uma explicação teórica e prática do algoritmo RSA, que constitui uma das técnicas mais confiáveis da área e que é amplamente utilizada em operações computacionais que lidem com informações sigilosas 
	\\ \indent A importância da criptografia na atualidade é incontestável, por estar presente em diversas situações do nosso cotidiano, mesmo que não percebamos. Desde uma simples troca de \emph{e-mails} até operações bancárias, são necessárias técnicas que garantam a integridade das informações contidas no processo em questão. Mostramos também uma análise do algoritmo que tenta descobrir a chave secreta e porque o RSA é considerado tão seguro.
	\\ \indent Procuramos demonstrar ao leitor a aplicação da teoria dos números descrevendo seu papel no algoritmo RSA. O algoritmo RSA conta com diversos artifícios e propriedades disponibilizados pela teoria dos números. A ironia é que, Leonhard Euler formalizou a teoria dos números há mais de 200 anos, demonstrou e introduziu diversas propriedades que são aplicadas atualmente nas diversas tecnologias da informação. A base do algoritmo RSA, que por sua vez forma o alicerce de toda a comunicação digital atual, foi fundamentada há mais de 200 anos. O próprio Euler jamais imaginaria que sua contribuição poderia gerar um impacto tão grande, mesmo que de forma discreta, em toda a sociedade contemporânea. Isso só nos faz refletir a importância da matemática e os frutos que esta disciplina milenar produz quando aplicada no desenvolvimento de tecnologias para o homem.
