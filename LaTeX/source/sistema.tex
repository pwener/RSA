\chapter[Funcionamento do Sistema]{Funcionamento do Sistema}
\label{chap:sistema}
	O sistema requisita um texto, que pode ser inserido via console, através do endereço de um arquivo ou de uma lista de exemplos pré-definidos. Tal informação será armazenada em um arquivo (\emph{text/exported.txt}).
	\\ \indent Depois disso, o sistema oferece a opção de inserir dois números primos para gerar as chaves públicas e privadas, que ao serem inseridos irão passar pelo teste de primalidade de Miller-Selfridge-Rabin (MSR). Esse por sua vez é um teste de natureza probabilística, que basicamente procura testemunhas contra a condição de primalidade. Se não encontrado, ele julga o número um possível primo. Tem uma margem de erro de 25 porcento, pode ser reduzida repetindo o passo o qual o algoritmo gera números aleatórios para um mesmo número. As chaves serão armazenadas em dois arquivos na pasta \emph{keys}.
	\\ \indent Após gerado as chaves, o usuário poderá pedir para que o texto informado seja encriptado. Esse texto codificado será gravado em um arquivo (\emph{messages/encrypted.txt}). Para descriptar o texto, o usuário pode selecionar um arquivo externo ou deixar a opção padrão que é a referente ao arquivo que acabou de ser dito. O texto descriptado será gravado em um arquivo (\emph{messages/decrypted.txt}).
	\\ \indent O sistema também é capaz de buscar a chave privada através do método de força bruta, o qual tenta todas as chaves possíveis, comparando o resultado de cada chave com um dicionário de palavras pré-definidas, contidas no arquivo (\emph{dicionario.pt.txt}). Tal algoritmo possui um tempo de verificação muito elevado, se tornando um método muito ineficiente para chaves de grandes valores. Entretanto, um curioso fato foi verificado que um mesmo código pode ser quebrado por diferentes chaves, logo se concluiu a não unicidade de uma chave.